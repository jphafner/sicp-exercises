\documentclass[]{amsart}

\begin{document}

\section{Proof by Induction}

We start with the following definitions
\begin{align}
    \phi    &= \frac{1 + \sqrt{5}}{2} \\
    \psi    &= \frac{1 - \sqrt{5}}{2}
\end{align}
and are asked to prove that
\begin{equation}
    \mathrm{fib}(n) = \frac{\phi^n - \psi^n}{\sqrt{5}}
\end{equation}

For a proof by induction we must first prove that the trivial case is true, this is a trivial task.
Next we must show that if we have $n$, that $n+1$ is guaranteed to work.
We will show this by establishing that
\begin{equation}
    \mathrm{fib}(n+1) = \mathrm{fib}(n) +  \mathrm{fib}(n-1)
\end{equation}

The binomial expansion states that
\begin{equation}
    \left(1 + a\right) = \sum_{k=0}^{n} \left(\begin{array}{c} n \\ k \end{array} \right) a^k
\end{equation}

We rewrite  $\phi$ and $\psi$ accordingly
\begin{align}
    \left(2\phi\right)^n    &= \sum_{k=0}^{n} \left(\begin{array}{c} n \\ k \end{array} \right) a^k \\
    \left(2\psi\right)^n  &= \sum_{k=0}^{n} \left(\begin{array}{c} n \\ k \end{array} \right) (-1)^{k} a^k
\end{align}

where $a = \sqrt{5}$, to ease reading.

The difference between $\phi$ and $\psi$ are the terms that are negative in $\psi$.
\begin{equation}
    \phi^n-\psi^n = \left(\dfrac{1}{2}\right)^n \sum_{k=1,odd}^{n} \left(\begin{array}{c} n \\ k \end{array} \right) a^k
\end{equation}
We can then redefine our fib function as
\begin{equation}
    \mathrm{fib}(n) = \left(\dfrac{1}{2}\right)^n \sum_{k=1,odd}^{n} \left(\begin{array}{c} n \\ k \end{array} \right) a^{k-1}
\end{equation}

Now with the combination identity of
\begin{equation}
    \left(\begin{array}{c} n \\ k \end{array} \right) = \frac{n!}{k!(n-k)!}
\end{equation}



It is trivial now to establish that the case of $n$ and $n-1$ will sum to the case for $n+1$.
This is a property of the binomial theorem.


\end{document}


